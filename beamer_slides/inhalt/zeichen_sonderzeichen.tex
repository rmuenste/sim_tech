\begin{frame}
	\frametitle{Spezielle Zeichen}
	\begin{itemize}
		\item bei Verwendung des ASCII-Zeichensatzes: Buchstaben ohne Umlaute, Zahlen und einige Sonderzeichen
		\item manche Zeichen sind \LaTeX-Steuerzeichen und daher reserviert (\$, \_, \{, \},\textbackslash )
        \item solche Sonderzeichen können durch \textbackslash{} maskiert werden: \\[0.5cm]
		\begin{center}
			\begin{tabular}{|cc|cc|} \hline
				\$ & \textbackslash\$ & \% & \textbackslash\% \\
				\{ & \textbackslash\{ & \} & \textbackslash\} \\
				\# & \textbackslash\# & \_ & \textbackslash\_ \\
				\& & \textbackslash\& & & \\ \hline
			\end{tabular}
		\end{center}
	\end{itemize}
\end{frame}

\begin{frame}[fragile]
	\frametitle{Deutsche Texte - Sprachpakete}
    Ohne weitere Angaben nimmt \LaTeX{} an, dass der eingegebene Text in englischer Sprache ist. Daher muss ggf. ein zusätzliches Sprachpaket eingebunden werden:
	\begin{center}
		\begin{block}{Beispiel-Header: deutsches Sprachpaket}
			\begin{lstlisting}
\documentclass[a4paper]{article}      % DIN-A4 Papierformat
\usepackage[ngerman]{babel}           % deutsche Benennung
\usepackage[utf8]{inputenc}
\begin{document}
  ...
\end{document}
			\end{lstlisting}
		\end{block}
	\end{center} \vspace{-1cm}
	\begin{itemize}
		\item \keyword{babel} sorgt für Unterstützung anderer Sprachen (Formate, Umlaute, Benennungen, Silbentrennung)
		\item \keyword{inputenc} unterstützt die direkte Eingabe von Zeichen über die Tastatur
	\end{itemize}
\end{frame}


\begin{frame}[fragile]
	\frametitle{Deutsche Texte -- Umlaute und Anführungszeichen}
	
	Eingabe deutscher Texte:
	\begin{itemize}
		\item Umlaute: \befehl{"a}, \befehl{"o}, \befehl{"u}, \befehl{ss} für ä, ö, ü, ß
		\item Anführungszeichen: \befehl{glq}, \befehl{grq} bzw. \befehl{glqq}, \befehl{grqq} für \glq einfache\grq~ bzw. \glqq doppelte\grqq~ Anführungszeichen
	\end{itemize}
	\vfill
	
	\latexBeispielDirekt{Beispiel: deutsche Umlaute}{examples/Deutsche_Umlaute/Deutsche_Umlaute}
	\vfill
\end{frame}

\begin{frame}[fragile]
	\frametitle{Deutsche Texte - Silbentrennung}
	\begin{itemize}
		\item erfolgt automatisch
		\item mögliche Trennstellen können  durch \befehl{-} auch angegeben werden, z. B.\\
		\lstinline$Donau\-dampf\-schiff\-fahrts\-gesell\-schaft$\\
		oder für das gesamte Dokument in der Präambel:
		\lstinline$\hypenation{Donau\-dampf\-schiff\-fahrts\-gesell\-schaft}$
	\end{itemize}

\end{frame}
