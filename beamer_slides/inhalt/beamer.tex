\begin{frame}[fragile]
	\frametitle{\LaTeX\ Beamer}
	\begin{itemize}
		\item Klasse zur Erstellung von Präsentationen mit \LaTeX
		\item Benutzerhandbuch zur Klasse:
		\href{http://ftp.fau.de/ctan/macros/latex/contrib/beamer/doc/beameruserguide.pdf}{http://ftp.fau.de/ctan/macros/latex/contrib/beamer/doc/beameruserguide.pdf} \\[0.5cm]
		\item Grundkonzept: eine \emphkeyword{frame} entspricht einer Seite (Folie)
		\item innerhalb der Frames können (fast alle) normalen \LaTeX-Befehle verwendet werden
	\end{itemize}
\end{frame}

\begin{frame}[fragile]
	\frametitle{\LaTeX\ Beamer -- Präambel}
	\begin{block}{\LaTeX\ Beamer -- Beispiel - Präambel}
		\begin{lstlisting}
\documentclass{beamer}
\usepackage[ngerman]{babel}
\usepackage[utf8]{inputenc}

\usetheme{Luebeck}
\usecolortheme{orchid}
\usefonttheme{default}
\useinnertheme{rounded}
\useoutertheme{shadow}
		\end{lstlisting}
	\end{block}
\end{frame}


\begin{frame}
	\frametitle{\LaTeX\ Beamer -- Hello World}
	\latexBeispielDatei{Hello World Frame}{examples/Beamer_HelloWorld/beamer_helloworld}
\end{frame}

\begin{frame}[fragile]
	\frametitle{\LaTeX\ Beamer -- Frames}
	
	\begin{block}{\LaTeX Beamer -- Frames -- Allgemeine Form}
		\begin{lstlisting}
\begin{frame}[Overlay][Optionen]{Titel}{Untertitel}
  Inhalt
\end {frame}
		\end{lstlisting}
	\end{block}
	\begin{center}
		\begin{tabular}{rl}
			\emphkeyword{Overlay} & \keyword{<+->} sorgt dafür, dass Listen schrittweise aufgebaut werden\\
			\emphkeyword{Optionen} & Anzeigeoptionen für diese Folie (mehrere Optionen müssen\\
			& durch Kommas abgetrennt werden)\\
			\emphkeyword{Titel} & Titel der Folie; kann auch mit \befehl{frametitle} gesetzt werden\\
			\emphkeyword{Untertitel} & Untertitel der Folie; kann auch mit \befehl{framesubtitle}\\
			& gesetzt werden
		\end{tabular}
	\end{center}
\end{frame}

\begin{frame}[fragile]
	\frametitle{\LaTeX\ Beamer -- Frames -- Optionen}
	\begin{center}
		\begin{tabular}{rl}
			\emphkeyword{t} & Ausrichtung des Inhalts oben (\textit{top})\\
			\emphkeyword{c} & Ausrichtung des Inhalts mittig (\textit{center})\\
			\emphkeyword{b} & Ausrichtung des Inhalts unten (\textit{bottom})\\
			\emphkeyword{label=name} & Label für die Folie setzen\\
			\emphkeyword{plain} & Kopf- und Fußzeile unterdrücken\\
			\emphkeyword{squeeze} & Inhalt zusammenrücken\\
			\emphkeyword{fragile} & notwendig für Folien mit Quelltext (\keyword{verbatim})
		\end{tabular}
	\end{center}
\end{frame}

\begin{frame}[fragile]
	\frametitle{\LaTeX\ Beamer -- Blöcke}
	\begin{block}{Block}
		erzeugt durch:
		\begin{lstlisting}
			\begin{block}{Block}
			...
			\end{block}
		\end{lstlisting}
	\end{block}
	\begin{exampleblock}{Beispiel}
		erzeugt durch \keyword{exampleblock}-Umgebung
	\end{exampleblock}
	\begin{alertblock}{Wichtig}
		erzeugt durch \keyword{alertblock}-Umgebung
	\end{alertblock}
	\vfill
	\begin{itemize}
		\item nützlich um thematisch zusammenzufassen
		\item das Aussehen variiert je nach Themenvorlage und eigenen Einstellungen
	\end{itemize}
\end{frame}

\begin{frame}[fragile]
	\frametitle{\LaTeX\ Beamer -- Spalten}
	\vspace{-1cm}
	\begin{columns}
		\column{0.5\textwidth}{
			\begin{block}{Spalte 1}
				...
			\end{block}}
		\column{0.5\textwidth}{\begin{block}{Spalte 2}
				...
			\end{block}}
	\end{columns}
	\vfill
	\begin{block}{Quellcode}
		\begin{lstlisting}
\begin{columns}
  \column{0.5\textwidth}{
    \begin{block}{Spalte 1}
      ...
    \end{block}}
  \column{0.5\textwidth}{
    \begin{block}{Spalte 2}
      ...
    \end{block}}
\end{columns}
		\end{lstlisting}
	\end{block}
	\begin{itemize}
		\item auch mehr als zwei Spalten möglich
	\end{itemize}
\end{frame}

\begin{frame}[fragile]
	\frametitle{\LaTeX\ Beamer -- Overlays}
	\begin{itemize}
		\item Frames können mehrere Overlays enthalten.
		\item Overlays sorgen dann für das stückweise "Aufbauen" einer Folie.
		\item Der \LaTeX-Seitenzähler wird dabei angehalten.
		\item Inhalte können nach und nach Erscheinen oder nur zu bestimmten Zeiten sichtbar sein.
	\end{itemize}
	
	\begin{columns}
		\column{0.5\textwidth}{
			\begin{block}{Quelltext}
				\begin{lstlisting}
Erster Teil
\pause \\
Zweiter Teil
				\end{lstlisting}
			\end{block}}
		\column{0.5\textwidth}{\begin{block}{Vorschau}
				Erster Teil
				\pause \\
				Zweiter Teil
			\end{block}}
	\end{columns}
\end{frame}

\begin{frame}[fragile]
	\frametitle{\LaTeX\ Beamer -- Overlays -- Beispiele Quellcode}
	\begin{block}{Beispiele für Overlay Steuerung}
		\begin{lstlisting}
\begin{itemize}
  \visible<1>{\item Dieser Text erscheint nur auf Overlay 1.}
  {\color<1-3>{red}{\item Dieser Text ist auf Overlays 1 bis 3 rot.}}
  {\color<2->{blue}{\item Dieser Text ist ab Overlay 2 blau.}}
  \only<-3>{\item Dieser Text erscheint nur bis Overlay 3.}
  \textbf<1,3,5>{\item Dieser Text erscheint auf Overlays 1, 3 und 5 im Fettdruck.}
  \alt<2>{\item Dieser Text erscheint nur auf Overlay 2.}{\item Sonst erscheint dieser Text.}
\end{itemize}
		\end{lstlisting}
	\end{block}
\end{frame}

\begin{frame}[fragile]
	\frametitle{\LaTeX\ Beamer -- Overlays -- Beispiele Quellcode}
	\begin{flushright}
		Overlay \only<1>{1}\only<2>{2}\only<3>{3}\only<4>{4}\only<5>{5} / 5
	\end{flushright}
	\begin{itemize}
		\visible<1>{\item Dieser Text erscheint nur auf Overlay 1.}
		{\color<1-3>{red}{\item Dieser Text ist auf Overlays 1 bis 3 rot.}}
		{\color<2->{blue}{\item Dieser Text ist ab Overlay 2 blau.}}
		\only<-3>{\item Dieser Text erscheint nur bis Overlay 3.}
		\textbf<1,3,5>{\item Dieser Text erscheint auf Overlays 1, 3 und 5 im Fettdruck.}
		\alt<2>{\item Dieser Text erscheint nur auf Overlay 2.}{\item Sonst erscheint dieser Text.}
	\end{itemize}
\end{frame}