\begin{frame}[fragile]
	\frametitle{Theorem-Umgebung}
	\begin{itemize}
		\item der \befehl{newtheorem} Befehl dient der Erzeugung von Umgebungen für Theoreme, Sätze, Definitionen etc.
		\begin{itemize}
			\item \befehl{newtheorem\{name\}\{beschriftung\}}
			\item \befehl{newtheorem\{name\}\{beschriftung\}[zaehler]}
		\end{itemize}
		\begin{center}
			\begin{tabular}{rl}
				\emphkeyword{name} & Name der Theorem-Umgebung\\
				\emphkeyword{beschriftung} & die Bezeichnung der Umgebung im Dokument\\
				\emphkeyword{zaehler} & der Zähler der zur Nummerierung verwendet wird
			\end{tabular}
		\end{center}
	\end{itemize}
\end{frame}

\begin{frame}
	\frametitle{Theorem-Umgebung -- Beispiel}
	
	\latexBeispielDirekt{Theorem-Umgebung -- Beispiel}{examples/Theorem_Umgebung/Theorem_Umgebung}
\end{frame}